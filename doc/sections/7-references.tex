\section{Appendices}
\label{sec:appendices_inst}

Appendices can be specified using \verb|\appendix|.
For example, our Troubleshooting instructions in
\iflabelexists{appendix:troubleshooting}
  {\cref{appendix:troubleshooting}}
  {the appendix of the full paper at \url{https://osf.io/nrmyc}}.

Note that the paper submission has to end after the \textbf{References} section and within the page limit of the conference you are submitting to.
Any version of Appendices or the paper with Appendices included has to be submitted separately as supplementary material.
You can use the \verb|hideappendix| class option to remove everything after \verb|\appendix|.
We encourage you to submit a full version of your paper to a preprint server with any appendices included.

You can use the \verb|\iflabelexists| macro to cross reference an appendix from the main text, but only if that label (i.e.\ the appendix) actually exists.
For example, above we use 

\begin{verbatim}
\iflabelexists{appendix:troubleshooting}
  {\cref{appendix:troubleshooting}}
  {the appendix of the full paper at
   \url{https://osf.io/XXXXX}}.
\end{verbatim}

in order to cross-reference to the appendix with \verb|\cref| if it exists, but if the appendix is commented out then we will simply create a hyperlinked URL to it.
