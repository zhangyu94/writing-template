\section{Figures}

\subsection{Loading figures}

The style automatically looks for image files with the correct extension (eps for regular \LaTeX; pdf, png, and jpg for pdf\LaTeX), in a set of given subfolders defined above using \verb|\graphicspath|: figures/, pictures/, images/.
It is thus sufficient to use \verb|\includegraphics{CypressView}| (instead of \verb|\includegraphics{pictures/CypressView.jpg}|).
Figures should be in CMYK or Grey scale format, otherwise, colour shifting may occur during the printing process.

\subsection{Vector figures}

Vector graphics like svg, eps, pdf are best for charts and other figures with text or lines.
They will look much nicer and crisper and any text in them will be more selectable, searchable, and accessible.

\subsection{Raster figures}

Of the raster graphics formats, screenshots of user interfaces and text, as well as line art, are better shown with png.
jpg is better for photographs.
Make sure all raster graphics are captured in high enough resolution so they look crisp and scale well.

\subsection{Alt texts}

Add alternative texts that describe the content of the image to all figures.

\subsection{Figures on the first page}

The teaser figure should only have the width of the abstract as the template enforces it.
The use of figures other than the optional teaser is not permitted on the first page.
Other figures should begin on the second page.
Papers submitted with figures other than the optional teaser on the first page will be refused.

\subsection{Subfigures}

You can add subfigures using the \texttt{subcaption} package that is automatically loaded.
Inside a \verb|figure| environment, create a \verb|subfigure| environment.
See \cref{fig:ex_subfigs} for an example.
You can reference individual figures, either fully using \verb|\cref| (\cref{fig:ex_subfigs_a,fig:ex_subfigs_b}) or by letter using \verb|\subref|.
E.g., \subref{fig:ex_subfigs_b}, \subref{fig:ex_subfigs_c}.
Note that \verb|\subref| only works for one label at a time.

\begin{figure}[tbp]
  \centering
  \begin{subfigure}[b]{0.45\columnwidth}
  	\centering
  	\includegraphics[width=\textwidth, alt={Big letter A on a gray background.}]{example-image-a}
  	\caption{The letter A.}
  	\label{fig:ex_subfigs_a}
  \end{subfigure}%
  \hfill%
  \begin{subfigure}[b]{0.45\columnwidth}
  	\centering
  	\includegraphics[width=\textwidth, alt={Big letter B on a gray background.}]{example-image-b}
  	\caption{The letter B.}
  	\label{fig:ex_subfigs_b}
  \end{subfigure}%
  \\%
  \begin{subfigure}[b]{0.45\columnwidth}
  	\centering
  	\includegraphics[width=\textwidth, alt={Big letter C on a gray background.}]{example-image-c}
  	\caption{The letter C.}
  	\label{fig:ex_subfigs_c}
  \end{subfigure}%
  \subfigsCaption{Example of adding subfigures with the \texttt{subcaption} package.}
  \label{fig:ex_subfigs}
\end{figure}

\subsection{Figure Credits}
\label{sec:figure_credits_inst}

In the \hyperref[sec:figure_credits]{Figure Credits} section at the end of the paper, you should credit the original sources of any figures that were reproduced or modified.
Include any license details necessary, as well as links to the original materials whenever possible.
For credits to figures from academic papers, include a citation that is listed in the \textbf{References} section.
An example is provided \hyperref[sec:figure_credits]{below}.
